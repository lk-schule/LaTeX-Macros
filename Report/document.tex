\documentclass[ngerman,12pt,titlepage]{scrartcl}

\usepackage[ngerman]{babel}

\usepackage[T1]{fontenc}
\usepackage[utf8]{inputenc}
\usepackage{varioref}
\usepackage{hyperref}
\usepackage{cleveref}
\usepackage{makeidx}
\usepackage{listings}
\usepackage[style=authoryear,backend=biber, natbib=true, hyperref=true]{biblatex}
\addbibresource{mybib.bib}
\makeindex
\usepackage{graphicx}
\usepackage{listings}

\title{Makros und deren Nutzen im alltäglichen Gebrauch von \LaTeX}
\author{Luca Kiebel}
\date{\today}


\begin{document}
\maketitle
\newpage
	\tableofcontents
\newpage

\section{Einleitung}
\label{cha:Einleitung}
\subsection{Was sind Makros?}
\textbf{\textit{Ein Makro ist in der Softwareentwicklung eine unter einer bestimmten Bezeichnung (Makroname) zusammengefasste Folge von Anweisungen oder Deklarationen, um diese mit nur einem einfachen Aufruf ausführen zu können.}} ~\citep{wiki:makro}\\ \\
Durch Makros lassen sich also lange Befehlsketten in kürzere umwandeln. Die ersten Makros 	\\ \\
Es gibt verschiedene Arten von Makros: \\
\vspace{5mm} \\
\begin{tabular}{l|p{0.65\textwidth}}
	\textbf{Art} & \textbf{ Beschreibung} \\ \hline
	Tastatur und Maus Makros & Kurze Abfolgen von Tastenanschlägen werden in längere transformiert. Häufig verwendete oder sich wiederholende Sequenzen von Tastenanschlägen und Mausbewegungen können automatisiert werden. \\
	Programm-Makros und Skripting & Durch Programm-Makros können (z.B. durch Tastenkombinationen) lange Befehlsketten durch eine Operation gesteuert werden. \\
\end{tabular}


\subsection{Wofür werden Makros in \LaTeX  genutzt?}
In \LaTeX  können Makros genutzt werden{,} um zu verhindern{,} dass Code mehrfach geschrieben wird. Soll ein integrierter Befehl zum Beispiel immer mit den selben Parametern ausgeführt werden{,} lohnt es sich{,} dafür ein Makro anzulegen. Auch{,} kann man neue Befehle anlegen, diese kurz machen{,} um Zeichen im Dokument zu sparen.


\newpage
\printbibliography

\end{document}
